\section{Applications}
    
    By identifying numeric characteristics of tandem fin wakes that are observable with a lateral line and relating these characteristics to the fins' phase difference, this project will lay the groundwork for a feedback control loop. 
    
    Tandem fins offer an incredible increase in maneuverability relative to single fin designs, and varying the phase difference between the fins can create nearly doubled thrust \citep{Gopalkrishnan1994, Muscutt2017}. However, in the design of bio-inspired vehicles, open loop programming requires slow, intense calculations of the fluid dynamics around the vehicle and the result is not robust in the face of environmental disturbances, such as collisions, object wakes, and cross currents. A passive feedback mechanism able to estimate the vortex pairing in real time and adjust the foil oscillation phase difference would improve thrust in the presence of disturbances, while also simplifying onboard vehicle trajectory calculations.
    
    The ability to infer upstream behaviors based on observed wake characteristics enables underwater schooling and wake tracking, behaviors that are quite desirable for UUV design. This is especially attractive for naval applications, and the Office of Naval Research is currently funding bio-inspired sensing methods under Code 34.
    
    While previous teams have used artificial lateral lines to detect flow features from external sources, such as obstacles and vortex-producing shapes, this project will be the first work that employs a lateral line in detecting self-generated fluid interactions. This research is novel and holds the potential for significant future experimentation and implementation.
    
    Future teams can build on this research by examining the effects of heave motions, flexible fins, and 3D wakes on the vortex interactions seen by the lateral line. The lateral line built for this project will also enable future USNA teams to investigate 'touch at a distance' for naval vessels.