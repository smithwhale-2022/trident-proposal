\section{Risk Management}
    This project has a high likelihood of results due to its simplifications, observed behaviors in nature, and readily available materials.
    
\subsection{Simplifications}
    While the tandem fin system is inspired by biological systems, many of the factors that complicate its study are omitted, which makes the results more consistent and attainable within the Trident timeline. In nature, fins are flexible, have complex 3D shapes, and move with both pitch and heave motions. This project utilizes rigid, 2D foils that only make pitch motions. Previous work has shown that this arrangement still achieves improved thrust and power compared to single fin arrangements, while also creating measurable vortex wakes. These simplifications allow this project to use USNA-built fins and previous foil actuation mechanisms developed in previous projects. This requires less time to be spent on experimental setup, and this time will be reallocated for signal processing.
    
\subsection{Readily Available Materials}
    This project's methods will be easily attainable using USNA's existing resources. All experiments will be conducted in the Large Re-Circulating Water Tunnel, which should not need significant modification before use. The tandem foils can be easily built in the Model Shop using foam and aluminum, but, in the case of setbacks, they can also be made with 3D printers or ordered online in slightly different shapes. The foil actuation mechanisms can utilize materials already under development in the Biomechanics Laboratory. Meanwhile, the lateral line will use sensors that are relatively inexpensive, and the Hydromechanics Laboratory staff has a high degree of familiarity with their use. The data processing tools that will likely be used for this project are readily available in MATLAB with online documentation. The majority of the supplies and expertise for this project's materials and methods are located at USNA, making this project robust against supply chain disturbances and experimental roadblocks.
        
\subsection{Existence of Results}
    As discussed in Section \ref{Natural Systems}, fish have been observed using their lateral lines to achieve desired thrust, power, and efficiency. They sense wake features and adjust their gaits accordingly, demonstrating a real time feedback control loop based on observable wake features. Previous researchers have characterized the wake interactions of tandem foils and the ability for current lateral line technology to determine wake features. The missing link between observed features and tandem fin wake interaction has already been observed in nature, showing that this project's solution exists.