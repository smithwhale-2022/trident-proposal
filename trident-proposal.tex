\documentclass[10pt]{article}

\title{Proposal title here}
\author{MIDN 2/C Cameron Smith\thanks{Department of Weapons, Robotics, and Control Engineering at the United States Naval Academy. Address for correspondence \emph{m226072@usna.edu}}}
\date{\today}

\usepackage[round,authoryear]{natbib}
\bibliographystyle{apalike}

\begin{document}
\maketitle

\begin{abstract}
The lateral line structures in fish contain two types of sensors: the surface neurotransmitters, which detect slow, even flow, and canal neurotransmitters (CNs), which detect changing
%Is "neurotransmitter" the right word here or is it like neuromast or something
flow. In \citep{klein2011title} these CNs were built artificially with small fiber optic strands that were watched by a light sensor as water flowed past them in a small channel. When the cant of the strands changed, the amount of light seen by the sensor either increased or decreased, producing an analog voltage change. As shown in a separate Klein experiment \citep{klein2011title}, these sensors were able to correctly indicate the frequency of vortex shedding from a cylinder downstream of the sensors. I propose that these sensors can be replicated and incorporated into a rigid fluke in order to
%first thing a biologist thinks is why rigid nothing is rigid
better understand the complex vorticity induced on it by upstream fin arrangements. The first phase of this experiment will be to build the sensors themselves and a rigid fluke body to embed them in. The second phase will be to validate the sensors by incorporating them into a line at the center of the fluke and exposing them to vortex shedding at a known frequency from a cylinder. Last, they will be installed at several different points in the fluke and an artificial dorsal fin will be placed upstream. When the system is run, the dorsal fin will produce a complex flow over the fluke that has vortical properties (need to find good source here), and the frequency output at the different sensors will indicate the frequency of vortices experienced at these different points in the fluke, while their magnitude will indicate the strength of these vortices.
\end{abstract}
% anatomy wise - dorsal fin on a whale is at 90 deg to the 
% tail flukes.. previous usna work dealt with humpback whale
% pectoral fin which does not have an upstream thing 


{\scriptsize Keywords: flapping foil, sensing, biomechanics, hydrodynamics, maneuvering}

\bibliography{trident.bib}
\end{document}

%
%
%Would any of you be available to meet tomorrow, December 19, to discuss project direction? I have been working on the background and previous work segments of my proposal, and before I begin hammering away at my objectives, I would like to run my goals by you all to see if they're reasonable. I've also come across some concepts that may refine my direction. Some groups have used simple pressure sensors to calculate the frequencies and magnitudes present for vortex wakes, while other research has been directed at showing how foils flapping in tandem affect the wake produced. When foils are flapping a certain phase for maximum thrust, their vortex streets sync and build constructively, but when they're out of the right phase, their vortices either build destructively or don't influence each other at all, creating little benefit on thrust or efficiency. I see a possible application for the lateral line sensors here, as they could be used as closed loop sensors to find the amount tandem foils need to adjust themselves by to get back in phase. A sample abstract would be:
%
%Ray finned fishes use their dorsal fins to create small vortex wakes that build constructively with the vortices created by their caudal fins, causing an increase in either thrust or efficiency, depending on the phase of the caudal fin’s movement. Such a behavior is desirable in autonomous underwater vehicles (AUV) in order to increase agility and endurance. While several groups have experimented with building robotic fish with dorsal fins, their control algorithms have been entirely open loop, requiring complex calculations of fluid dynamics or flow visualization techniques in order to find the phase agreement of their flapping architectures. 
%
%Simultaneously, researchers have explored artificial replications of fish lateral lines as flow visualization tools. In the wild, the lateral line is used for sensing objects, wakes, and fluid flow characteristics, including vortices. Several teams have employed off-the-shelf pressure sensors to characterize the frequency and magnitude of vortex streets in a manner quite similar to a lateral line. These sensing arrays often use frequency-domain analysis with a fast Fourier transform (FFT) to calculate the frequency and magnitudes of vortex shedding.
%
%The goal of this research project is to utilize these pressure sensor-based lateral lines to create a real-time feedback loop that detects the agreeance between the vortex wakes created by two foils flapping in tandem. The novelty of this research would be the emergence of a wake characterization method that does not rely on the difficult calculations and assumptions required by fluid dynamics, as well as the lengthy tasks of digital particle image velocimetry (DPIV) which are often used for wake characterization. It would also provide a  real-time feedback controller for AUVs seeking to use tandem fin propulsion for increased efficiency.


%Good morning Professors,
%
%After reading through several more sources and deliberating, I believe I have a better defined project roadmap. I am also working on my proposal and would like to schedule another meet with you all when you are next available to touch base and talk about the way forward. How is Tuesday, December 1, at 0920? I am also available during lunch today and Tuesday, plus 1-3rd period on Wednesday.
%
%Places where this can be improved: There is a line of research that uses differential pressure gauges instead of artificial optical hair sensors, and these might be more practical because they’re ready made and easy to buy. Also, there is likely research on how to ‘tune’ CNs to specific frequencies by placing holes different distances apart from each other, and this could be incorporated into the construction phase of this project.




% DE: things to consider; sensing forces, sensing vibration,
% sensing if flow is attached, sensing lift/drag on fin, 
% sensing "luffing"... we've jumped straight to something about
% vorticity and circulation from an upstream body without
% explaining why; also look up stall sensing for wings, 
% GI taylor vorticity video wheere he makes little devices 
% track vorticity...

