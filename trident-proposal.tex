Good morning Professors,

After reading through several more sources and deliberating, I believe I have a better defined project roadmap. I am also working on my proposal and would like to schedule another meet with you all when you are next available to touch base and talk about the way forward. How is Tuesday, December 1, at 0920? I am also available during lunch today and Tuesday, plus 1-3rd period on Wednesday.

Here is what I have narrowed my project direction down to:

The lateral line structures in fish contain two types of sensors: the surface neurotransmitters, which detect slow, even flow, and canal neurotransmitters (CNs), which detect changing flow. In (Klein, 2011) these CNs were built artificially with small fiber optic strands that were watched by a light sensor as water flowed past them in a small channel. When the cant of the strands changed, the amount of light seen by the sensor either increased or decreased, producing an analog voltage change. As shown in a separate Klein experiment (Klein, 2011), these sensors were able to correctly indicate the frequency of vortex shedding from a cylinder downstream of the sensors. I propose that these sensors can be replicated and incorporated into a rigid fluke in order to better understand the complex vorticity induced on it by upstream fin arrangements. The first phase of this experiment will be to build the sensors themselves and a rigid fluke body to embed them in. The second phase will be to validate the sensors by incorporating them into a line at the center of the fluke and exposing them to vortex shedding at a known frequency from a cylinder. Last, they will be installed at several different points in the fluke and an artificial dorsal fin will be placed upstream. When the system is run, the dorsal fin will produce a complex flow over the fluke that has vortical properties (need to find good source here), and the frequency output at the different sensors will indicate the frequency of vortices experienced at these different points in the fluke, while their magnitude will indicate the strength of these vortices.


Places where this can be improved: There is a line of research that uses differential pressure gauges instead of artificial optical hair sensors, and these might be more practical because they’re ready made and easy to buy. Also, there is likely research on how to ‘tune’ CNs to specific frequencies by placing holes different distances apart from each other, and this could be incorporated into the construction phase of this project.
